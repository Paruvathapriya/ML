\documentclass[11pt]{article}
\usepackage[margin=1in]{geometry}
\usepackage{enumitem}
\usepackage{hyperref}
\usepackage{graphicx}
\usepackage{array}
\usepackage{multicol}
\usepackage{enumitem}
\usepackage{longtable}
\usepackage{titlesec}

\begin{document}

\begin{center}
    \large \textbf{Sri Sivasubramaniya Nadar College of Engineering, Chennai} \\
    (An autonomous Institution affiliated to Anna University) \\
    \vspace{0.3cm}
\end{center}
\begin{table}[!h]
\renewcommand{\arraystretch}{1.5}
\resizebox{\textwidth}{!}{%
\begin{tabular}{|l|cll|}
\hline
Degree \& Branch     & \multicolumn{1}{c|}{B.E.   Computer Science \& Engineering} & \multicolumn{1}{l|}{Semester}        & V                                        \\ \hline
Subject Code \& Name & \multicolumn{3}{c|}{ICS1512 \& Machine Learning Algorithms Laboratory}                                                                              \\ \hline
Academic year       & \multicolumn{1}{c|}{2025-2026 (Odd)}                        & \multicolumn{1}{c|}{Batch:2023-2028} & \multicolumn{1}{c|}{\textbf{Due date:}} \\ \hline
\end{tabular}%
}
\end{table}
\begin{center}
 \textbf{Experiment  2: Title Loan Amount Prediction using Linear Regression}
\end{center}
\textbf{Aim:}

\vspace{0.5cm}
\noindent
To develop a machine learning model using Linear Regression to predict the sanctioned loan amount based on historical applicant data, by preprocessing and analyzing the dataset, applying feature engineering techniques, and evaluating model performance using metrics such as MAE, MSE, RMSE, and R² score with appropriate visualizations.

\vspace{0.5cm}
\noindent
\textbf{\large Libraries Used:}

\begin{itemize}
  \item pandas
  \item numpy
  \item matplotlib.pyplot
  \item seaborn
  \item scikit-learn
\end{itemize}

\vspace{0.5cm}
\noindent
\textbf{Mathematical Description}

\vspace{0.5cm}
\noindent
In this experiment, Linear Regression is used to predict the loan sanction amount based on several input features.

\noindent
The mathematical model for Linear Regression is represented as:

\[
y = \beta_0 + \beta_1 x_1 + \beta_2 x_2 + \cdots + \beta_n x_n + \varepsilon
\]

\noindent
where:
\begin{itemize}
    \item $y$ is the dependent variable (Loan Sanction Amount),
    \item $x_1, x_2, \ldots, x_n$ are independent variables (features such as Age, Income, Credit Score, etc.),
    \item $\beta_0$ is the intercept,
    \item $\beta_1, \beta_2, \ldots, \beta_n$ are the coefficients of the features,
    \item $\varepsilon$ is the error term representing noise or unexplained variance.
\end{itemize}

\noindent
The coefficients $\beta$ are estimated by minimizing the Residual Sum of Squares (RSS), defined as:

\[
RSS = \sum_{i=1}^{m} \left( y_i - \hat{y}_i \right)^2 = \sum_{i=1}^{m} \left( y_i - \left( \beta_0 + \sum_{j=1}^{n} \beta_j x_{ij} \right) \right)^2
\]

\noindent
where $m$ is the number of observations. The model is evaluated using the following metrics:

\begin{itemize}
    \item \textbf{Mean Absolute Error (MAE)} – average absolute difference between actual and predicted values,
    \item \textbf{Mean Squared Error (MSE)} – average squared difference between actual and predicted values,
    \item \textbf{Root Mean Squared Error (RMSE)} – square root of MSE, providing error in original units,
    \item \textbf{R-squared ($R^2$)} – proportion of variance explained by the model,
    \item \textbf{Adjusted $R^2$} – adjusted for number of predictors to prevent overfitting.
\end{itemize}

\documentclass{article}
\usepackage[utf8]{inputenc}
\usepackage{geometry}
\usepackage{listings}
\usepackage{xcolor}

\geometry{margin=1in}

\definecolor{codegray}{gray}{0.95}
\definecolor{darkblue}{rgb}{0.1,0.1,0.6}

\lstdefinestyle{mystyle}{
    backgroundcolor=\color{codegray},
    commentstyle=\color{gray},
    keywordstyle=\color{blue},
    numberstyle=\tiny\color{gray},
    stringstyle=\color{darkblue},
    basicstyle=\ttfamily\footnotesize,
    breakatwhitespace=false,         
    breaklines=true,                 
    captionpos=b,                    
    keepspaces=true,                 
    numbers=left,                    
    numbersep=5pt,                  
    showspaces=false,                
    showstringspaces=false,
    showtabs=false,                  
    tabsize=2
}

\lstset{style=mystyle}

\begin{document}

\section*{Python Code}

\begin{verbatim}
import numpy as np
import pandas as pd
import matplotlib.pyplot as plt
from sklearn.linear_model import LinearRegression

df = pd.read_csv(r"C:/Users/madhu/Downloads/archive/train.csv")
df.head()
df.describe()
df.isnull().sum()

df["Gender"]=df["Gender"].fillna(df["Gender"].mode()[0])
df["Income (USD)"]=df["Income (USD)"].fillna(df["Income (USD)"].mean())
df["Income Stability"]=df["Income Stability"].fillna(df["Income Stability"].mode()[0])
df["Type of Employment"]=df["Type of Employment"].fillna("unknown")
df["Current Loan Expenses (USD)"]=df["Current Loan Expenses (USD)"].fillna(df["Income (USD)"].mean())
df["Dependents"]=df["Dependents"].fillna(df["Dependents"].mean())
df["Credit Score"]=df["Credit Score"].fillna(df["Credit Score"].mean())
df["Has Active Credit Card"]=df["Has Active Credit Card"].fillna(df["Has Active Credit Card"].mode()[0])
df["Property Age"]=df["Property Age"].fillna(df["Property Age"].mean())
df["Property Location"]=df["Property Location"].fillna(df["Property Location"].mode()[0])
df.dropna(subset=["Loan Sanction Amount (USD)"], inplace=True)

df.drop(columns=["Customer ID", "Name", "Property ID","Type of Employment","Profession"], inplace=True)

df["Gender"] = df["Gender"].map({"M": 1, "F": 0})
df["Expense Type 1"] = df["Expense Type 1"].map({"Y": 1, "N": 0})
df["Expense Type 2"] = df["Expense Type 2"].map({"Y": 1, "N": 0})
df["Has Active Credit Card"] = df["Has Active Credit Card"].map({
    "Active": 1,
    "Inactive": 0,
    "Unpossessed": -1
})
df["Income Stability"] = df["Income Stability"].map({"Low": 0, "High": 1})
df["Property Location"] = df["Property Location"].map({"Rural": 0, "Semi-Urban": 1, "Urban": 2})
df["Location"] = df["Location"].map({"Rural": 0, "Semi-Urban": 1, "Urban": 2})

x = df.drop('Loan Sanction Amount (USD)', axis=1)
y = df["Loan Sanction Amount (USD)"]

from sklearn.model_selection import train_test_split
x_train, x_test, y_train, y_test = train_test_split(x, y, test_size=0.3, random_state=42)

from sklearn.impute import SimpleImputer
imputer = SimpleImputer(strategy="mean")
x_train = imputer.fit_transform(x_train)
x_test = imputer.transform(x_test)

model = LinearRegression()
model.fit(x_train, y_train)
y_pred = model.predict(x_test)

from sklearn.metrics import mean_squared_error, r2_score
r2_error = r2_score(y_test, y_pred)
print(r2_error)
print(df.columns)

from sklearn.model_selection import KFold
from sklearn.preprocessing import StandardScaler

k = 5
kf = KFold(n_splits=k, shuffle=True, random_state=42)
model = LinearRegression()

mse_list = []
r2_list = []
rmse = []

for fold, (train_idx, test_idx) in enumerate(kf.split(x), 1):
    X_train, X_test = x.iloc[train_idx], x.iloc[test_idx]
    y_train, y_test = y.iloc[train_idx], y.iloc[test_idx]

    scaler = StandardScaler()
    X_train_scaled = scaler.fit_transform(X_train)
    X_test_scaled = scaler.transform(X_test)

    model.fit(X_train_scaled, y_train)
    y_pred = model.predict(X_test_scaled)

    mse = mean_squared_error(y_test, y_pred)
    r2 = r2_score(y_test, y_pred)
    rms = np.sqrt(mse)

    mse_list.append(mse)
    r2_list.append(r2)
    rmse.append(rms)

    print(f"Fold {fold}: MSE = {mse:.2f}, R2 = {r2:.2f}, RMSE = {rms:.2f}")

print(f"\nAverage MSE: {np.mean(mse_list):.2f}")
print(f"Average R2: {np.mean(r2_list):.2f}")
print(f"Average RMSE: {np.mean(rmse):.2f}")

plt.figure(figsize=(10, 5))
plt.plot(range(1, k+1), mse_list, marker='o', label='MSE per Fold')
plt.plot(range(1, k+1), r2_list, marker='s', label='R² per Fold')
plt.title("K-Fold Validation Performance")
plt.xlabel("Fold")
plt.ylabel("Score")
plt.legend()
plt.grid(True)
plt.show()

import seaborn as sns
sns.set(style="whitegrid")

categorical_cols = ['Gender', 'Income Stability', 'Location', 'Expense Type 1', 'Expense Type 2',
                    'Has Active Credit Card', 'Property Type', 'Property Location', 'Co-Applicant']

for col in categorical_cols:
    plt.figure(figsize=(8, 4))
    sns.countplot(data=df, x=col, palette="Set2")
    plt.title(f"Count Plot of {col}")
    plt.xticks(rotation=45)
    plt.tight_layout()
    plt.show()

target = 'Loan Sanction Amount (USD)'
numerical_cols = ['Age', 'Income (USD)', 'Loan Amount Request (USD)', 'Current Loan Expenses (USD)',
                  'Dependents', 'Credit Score', 'No. of Defaults', 'Property Age', 'Property Price']

for col in ['Income (USD)', 'Credit Score', 'Property Price', 'Loan Amount Request (USD)']:
    plt.figure(figsize=(6, 4))
    sns.scatterplot(x=col, y=target, data=df)
    plt.title(f"{target} vs {col}")
    plt.tight_layout()
    plt.show()

plt.figure(figsize=(10, 8))
corr = df[numerical_cols + [target]].corr()
sns.heatmap(corr, annot=True, cmap="coolwarm", fmt=".2f")
plt.title("Correlation Heatmap")
plt.tight_layout()
plt.show()

for col in ['Income (USD)', 'Loan Amount Request (USD)', target]:
    plt.figure(figsize=(6, 4))
    sns.boxplot(x=df[col])
    plt.title(f"Boxplot of {col}")
    plt.tight_layout()
    plt.show()

plt.figure(figsize=(6, 4))
sns.scatterplot(x=y_test, y=y_pred)
plt.xlabel("Actual Loan Amount")
plt.ylabel("Predicted Loan Amount")
plt.title("Actual vs Predicted")
plt.plot([y.min(), y.max()], [y.min(), y.max()], color='red', linestyle='--')
plt.tight_layout()
plt.show()

residuals = y_test - y_pred
plt.figure(figsize=(6, 4))
sns.scatterplot(x=y_pred, y=residuals)
plt.axhline(0, color='red', linestyle='--')
plt.xlabel("Predicted Values")
plt.ylabel("Residuals")
plt.title("Residual Plot")
plt.tight_layout()
plt.show()

coefficients = pd.Series(model.coef_, index=x.columns).sort_values()
plt.figure(figsize=(10, 8))
coefficients.plot(kind="bar")
plt.title("Feature Importance (Linear Coefficients)")
plt.tight_layout()
plt.show()

print("R2 Score:", r2_score(y_test, y_pred))
print("RMSE:", np.sqrt(mean_squared_error(y_test, y_pred)))

# Final Fold Loop Example (alternate)
kf = KFold(n_splits=5, shuffle=True, random_state=42)
model = LinearRegression()
mse_scores = []
r2_scores = []

for fold, (train_idx, test_idx) in enumerate(kf.split(x), 1):
    X_train, X_test = x.values[train_idx], x.values[test_idx]
    y_train, y_test = y.values[train_idx], y.values[test_idx]

    scaler = StandardScaler()
    X_train_scaled = scaler.fit_transform(X_train)
    X_test_scaled = scaler.transform(X_test)

    model.fit(X_train_scaled, y_train)
    y_pred = model.predict(X_test_scaled)

    mse = mean_squared_error(y_test, y_pred)
    r2 = r2_score(y_test, y_pred)

    mse_scores.append(mse)
    r2_scores.append(r2)

    print(f"Fold {fold}: MSE = {mse:.2f}, R² = {r2:.2f}")

print(f"\nAverage MSE: {np.mean(mse_scores):.2f}")
print(f"Average R²: {np.mean(r2_scores):.2f}")

plt.figure(figsize=(10, 5))
plt.plot(range(1, k+1), mse_scores, marker='o', label='MSE')
plt.plot(range(1, k+1), r2_scores, marker='s', label='R²')
plt.xlabel("Fold")
plt.ylabel("Score")
plt.title("K-Fold Cross Validation Performance")
plt.legend()
plt.grid(True)
plt.show()
\end{verbatim}
\usepackage{graphicx}
\usepackage{float} % For using [H] to fix image position
\begin{figure}[H]
    \centering
    \includegraphics[width=0.8\textwidth]{images/performance_metric.png}
    \caption{Evaluation Metrics: MSE, R\textsuperscript{2}, and RMSE across 5 folds}
    \label{fig:model-results}
\end{figure}
\begin{figure}[H]
    \centering
    \includegraphics[width=0.85\textwidth]{images/histogram.png}
    \caption{Histogram of Loan Sanction Amount (USD)}
    \label{fig:loan-histogram}
\end{figure}
\begin{figure}[H]
    \centering
    \includegraphics[width=0.85\textwidth]{images/boxplot.png}
    \caption{Boxplot of features}
    \label{fig:boxplot}
\end{figure}
\begin{figure}[H]
    \centering
    \includegraphics[width=0.85\textwidth]{images/heatmap.png}
    \caption{correlation}
    \label{fig:boxplot}
\end{figure}
\begin{figure}[H]
    \centering
    \includegraphics[width=0.85\textwidth]{images/actual_predicted.png}
    \caption{actual vs predicted}
    \label{fig:boxplot}
\end{figure}
\begin{figure}[H]
    \centering
    \includegraphics[width=0.85\textwidth]{images/loan_sanction_amount.png}
    \caption{Loan sanction amount vs Loan amount request}
    \label{fig:boxplot}
\end{figure}
\documentclass{article}
\usepackage[margin=1in]{geometry}
\usepackage{booktabs}
\usepackage{longtable}
\usepackage{caption}
\usepackage{array}

\begin{document}

\section*{Cross-Validation Results Table}

If you have used K-Fold Cross-Validation (e.g., with \( K = 5 \)), report the evaluation metrics for each fold in the table below.

\begin{table}[h!]
\centering
\caption{Cross-Validation Results (\( K = \_\_ \))}
\begin{tabular}{|c|c|c|c|c|}
\hline
\textbf{Fold} & \textbf{MAE} & \textbf{MSE} & \textbf{RMSE} & \textbf{R\textsuperscript{2} Score} \\
\hline
Fold 1 & -- & 1017292280.07 & 31895.02 & 0.55 \\
Fold 2 & -- & 974737027.27 & 31220.78 & 0.57 \\
Fold 3 & -- & 1012251604.48 & 31815.90 & 0.56 \\
Fold 4 & -- & 919543050.51 & 30323.97 & 0.62 \\
Fold 5 & -- & 994672908.56 & 31538.44 & 0.58 \\
\hline
\textbf{Average} & & & & \\
\hline
\end{tabular}
\end{table}

\vspace{1cm}

\section*{Results Summary Table}

Students are expected to fill in the following table based on their model’s performance and visualizations.

\begin{table}[h!]
\centering
\caption{Summary of Results for Loan Amount Prediction}
\begin{tabular}{|p{7cm}|p{7cm}|}
\hline
\textbf{Description} & \textbf{Student’s Result} \\
\hline
Dataset Size (after preprocessing) & 29660 \\
Train/Test Split Ratio & 70/30 \\
Feature(s) Used for Prediction & Index(['Gender', 'Age', 'Income (USD)', 'Income Stability', 'Location',
       'Loan Amount Request (USD)', 'Current Loan Expenses (USD)',
       'Expense Type 1', 'Expense Type 2', 'Dependents', 'Credit Score',
       'No. of Defaults', 'Has Active Credit Card', 'Property Age',
       'Property Type', 'Property Location', 'Co-Applicant', 'Property Price',
       'Loan Sanction Amount (USD)'],
      dtype='object') \\
Model Used & Linear Regression \\
Cross-Validation Used? (Yes/No) & Yes \\
If Yes, Number of Folds (K) & 5 \\
Reference to CV Results Table & Table 1 \\
Mean Absolute Error (MAE) on Test Set & \\
Mean Squared Error (MSE) on Test Set & 983699374.18\\
Root Mean Squared Error (RMSE) on Test Set & 31358.82\\
R\textsuperscript{2} Score on Test Set & 0.58 \\
Adjusted R\textsuperscript{2} Score on Test Set & 0.5472\\
Most Influential Feature(s) & [’Co-Applicant
 0’, ’Property Price’, ’Credit
 Score’]\\
Observations from Residual Plot &The residual plot shows a clear pattern with residuals de
creasing as predicted values increase, indicating model bias
 and heteroscedasticity. This suggests the linear model may
 not fully capture the relationship. \\
Interpretation of Predicted vs Actual Plot & The residual plot shows a clear pattern with residuals decreasing as predicted values increase, indicating model bias
 and heteroscedasticity. This suggests the linear model may
 not fully capture the relationship.\\
Any Overfitting or Underfitting Observed? & \\
If Yes, Brief Justification (e.g., training vs test error, residual patterns) & No significant overfitting or underfitting observed. The
 training and validation errors are comparable, and resid
uals do not show extreme patterns, indicating the model
 generalizes reasonably well. However, some bias at higher
 values suggests slight underfitting in that range.\\
\hline
\end{tabular}
\end{table}
\documentclass{article}
\usepackage[margin=1in]{geometry}
\usepackage{enumitem}
\usepackage{titlesec}
\titleformat{\section}[block]{\large\bfseries}{}{0em}{}

\begin{document}

\section*{8. Best Practices}

\begin{itemize}[leftmargin=1.5em]
    \item \textbf{Data Preprocessing:} Handle missing values carefully (drop or impute), and remove irrelevant columns such as IDs and names that do not contribute to prediction.
    
    \item \textbf{Feature Engineering:} Create new meaningful features (e.g., total income) and apply transformations (e.g., log transformation for skewed data) to improve model performance.
    
    \item \textbf{Scaling and Encoding:} Use scaling (e.g., \texttt{StandardScaler}) for numerical features and one-hot encoding for categorical features to prepare data for linear regression.
    
    \item \textbf{Train-Test Split:} Use proper splits (e.g., 80/20) and consider cross-validation to ensure the model generalizes well and to prevent overfitting.
    
    \item \textbf{Model Evaluation:} Use multiple metrics (MAE, MSE, RMSE, R\textsuperscript{2}) to assess different aspects of model performance.
    
    \item \textbf{Residual Analysis:} Analyze residual plots to detect model bias or heteroscedasticity and decide if further feature engineering or alternative models are needed.
\end{itemize}

\vspace{1em}

\section*{9. Learning Outcomes}

Through this experiment, I have:

\begin{itemize}[leftmargin=1.5em]
    \item Understood the full ML pipeline from data cleaning to model evaluation.
    \item Learned the importance of feature engineering and proper data preprocessing.
    \item Recognized signs of overfitting or underfitting via residual and prediction plots.
    \item Used cross-validation for robust model performance assessment.
\end{itemize}
a
\end{document}

\end{document}






                 

Results and Discussions:

\vspace{0.5cm}
\noindent
Learning Practices


\vspace{0.5cm}
\noindent
\section*{Expected GitHub Repository Structure and Content}

For \textbf{Experiment 1: Working with Python packages – Numpy, Scipy, Scikit-learn, Matplotlib}, students maintaining a GitHub repository are expected to organize their work in a clear and structured manner. The repository should begin with a detailed \texttt{README.md} file that outlines the objective of the experiment, the tools and libraries used, a summary of files included, and instructions for running the code. Within the repository, there should be a dedicated folder (e.g., \texttt{Experiment-1}) containing well-documented Jupyter notebooks or Python scripts demonstrating core operations in Numpy (array manipulation), Pandas (data handling and preprocessing), Scipy (mathematical and statistical functions), Scikit-learn (basic model training and evaluation), and Matplotlib (visualizations like histograms, scatter plots, and heatmaps). Additionally, students may include a \texttt{datasets} folder with small CSV files such as the Iris or Diabetes datasets, or reference external data sources like Kaggle or the UCI repository. A \texttt{screenshots} folder should contain visual outputs of their code execution. A \texttt{requirements.txt} file listing the Python packages and versions used is also recommended to ensure reproducibility. Overall, the GitHub repository should reflect a well-organized, self-explanatory workspace that documents their learning process and experiment results effectively.








\end{document}
